% Options for packages loaded elsewhere
\PassOptionsToPackage{unicode}{hyperref}
\PassOptionsToPackage{hyphens}{url}
%
\documentclass[
  man]{apa6}
\usepackage{amsmath,amssymb}
\usepackage{lmodern}
\usepackage{iftex}
\ifPDFTeX
  \usepackage[T1]{fontenc}
  \usepackage[utf8]{inputenc}
  \usepackage{textcomp} % provide euro and other symbols
\else % if luatex or xetex
  \usepackage{unicode-math}
  \defaultfontfeatures{Scale=MatchLowercase}
  \defaultfontfeatures[\rmfamily]{Ligatures=TeX,Scale=1}
\fi
% Use upquote if available, for straight quotes in verbatim environments
\IfFileExists{upquote.sty}{\usepackage{upquote}}{}
\IfFileExists{microtype.sty}{% use microtype if available
  \usepackage[]{microtype}
  \UseMicrotypeSet[protrusion]{basicmath} % disable protrusion for tt fonts
}{}
\makeatletter
\@ifundefined{KOMAClassName}{% if non-KOMA class
  \IfFileExists{parskip.sty}{%
    \usepackage{parskip}
  }{% else
    \setlength{\parindent}{0pt}
    \setlength{\parskip}{6pt plus 2pt minus 1pt}}
}{% if KOMA class
  \KOMAoptions{parskip=half}}
\makeatother
\usepackage{xcolor}
\usepackage{graphicx}
\makeatletter
\def\maxwidth{\ifdim\Gin@nat@width>\linewidth\linewidth\else\Gin@nat@width\fi}
\def\maxheight{\ifdim\Gin@nat@height>\textheight\textheight\else\Gin@nat@height\fi}
\makeatother
% Scale images if necessary, so that they will not overflow the page
% margins by default, and it is still possible to overwrite the defaults
% using explicit options in \includegraphics[width, height, ...]{}
\setkeys{Gin}{width=\maxwidth,height=\maxheight,keepaspectratio}
% Set default figure placement to htbp
\makeatletter
\def\fps@figure{htbp}
\makeatother
\setlength{\emergencystretch}{3em} % prevent overfull lines
\providecommand{\tightlist}{%
  \setlength{\itemsep}{0pt}\setlength{\parskip}{0pt}}
\setcounter{secnumdepth}{-\maxdimen} % remove section numbering
% Make \paragraph and \subparagraph free-standing
\ifx\paragraph\undefined\else
  \let\oldparagraph\paragraph
  \renewcommand{\paragraph}[1]{\oldparagraph{#1}\mbox{}}
\fi
\ifx\subparagraph\undefined\else
  \let\oldsubparagraph\subparagraph
  \renewcommand{\subparagraph}[1]{\oldsubparagraph{#1}\mbox{}}
\fi
\newlength{\cslhangindent}
\setlength{\cslhangindent}{1.5em}
\newlength{\csllabelwidth}
\setlength{\csllabelwidth}{3em}
\newlength{\cslentryspacingunit} % times entry-spacing
\setlength{\cslentryspacingunit}{\parskip}
\newenvironment{CSLReferences}[2] % #1 hanging-ident, #2 entry spacing
 {% don't indent paragraphs
  \setlength{\parindent}{0pt}
  % turn on hanging indent if param 1 is 1
  \ifodd #1
  \let\oldpar\par
  \def\par{\hangindent=\cslhangindent\oldpar}
  \fi
  % set entry spacing
  \setlength{\parskip}{#2\cslentryspacingunit}
 }%
 {}
\usepackage{calc}
\newcommand{\CSLBlock}[1]{#1\hfill\break}
\newcommand{\CSLLeftMargin}[1]{\parbox[t]{\csllabelwidth}{#1}}
\newcommand{\CSLRightInline}[1]{\parbox[t]{\linewidth - \csllabelwidth}{#1}\break}
\newcommand{\CSLIndent}[1]{\hspace{\cslhangindent}#1}
\ifLuaTeX
\usepackage[bidi=basic]{babel}
\else
\usepackage[bidi=default]{babel}
\fi
\babelprovide[main,import]{english}
% get rid of language-specific shorthands (see #6817):
\let\LanguageShortHands\languageshorthands
\def\languageshorthands#1{}
% Manuscript styling
\usepackage{upgreek}
\captionsetup{font=singlespacing,justification=justified}

% Table formatting
\usepackage{longtable}
\usepackage{lscape}
% \usepackage[counterclockwise]{rotating}   % Landscape page setup for large tables
\usepackage{multirow}		% Table styling
\usepackage{tabularx}		% Control Column width
\usepackage[flushleft]{threeparttable}	% Allows for three part tables with a specified notes section
\usepackage{threeparttablex}            % Lets threeparttable work with longtable

% Create new environments so endfloat can handle them
% \newenvironment{ltable}
%   {\begin{landscape}\centering\begin{threeparttable}}
%   {\end{threeparttable}\end{landscape}}
\newenvironment{lltable}{\begin{landscape}\centering\begin{ThreePartTable}}{\end{ThreePartTable}\end{landscape}}

% Enables adjusting longtable caption width to table width
% Solution found at http://golatex.de/longtable-mit-caption-so-breit-wie-die-tabelle-t15767.html
\makeatletter
\newcommand\LastLTentrywidth{1em}
\newlength\longtablewidth
\setlength{\longtablewidth}{1in}
\newcommand{\getlongtablewidth}{\begingroup \ifcsname LT@\roman{LT@tables}\endcsname \global\longtablewidth=0pt \renewcommand{\LT@entry}[2]{\global\advance\longtablewidth by ##2\relax\gdef\LastLTentrywidth{##2}}\@nameuse{LT@\roman{LT@tables}} \fi \endgroup}

% \setlength{\parindent}{0.5in}
% \setlength{\parskip}{0pt plus 0pt minus 0pt}

% Overwrite redefinition of paragraph and subparagraph by the default LaTeX template
% See https://github.com/crsh/papaja/issues/292
\makeatletter
\renewcommand{\paragraph}{\@startsection{paragraph}{4}{\parindent}%
  {0\baselineskip \@plus 0.2ex \@minus 0.2ex}%
  {-1em}%
  {\normalfont\normalsize\bfseries\itshape\typesectitle}}

\renewcommand{\subparagraph}[1]{\@startsection{subparagraph}{5}{1em}%
  {0\baselineskip \@plus 0.2ex \@minus 0.2ex}%
  {-\z@\relax}%
  {\normalfont\normalsize\itshape\hspace{\parindent}{#1}\textit{\addperi}}{\relax}}
\makeatother

\makeatletter
\usepackage{etoolbox}
\patchcmd{\maketitle}
  {\section{\normalfont\normalsize\abstractname}}
  {\section*{\normalfont\normalsize\abstractname}}
  {}{\typeout{Failed to patch abstract.}}
\patchcmd{\maketitle}
  {\section{\protect\normalfont{\@title}}}
  {\section*{\protect\normalfont{\@title}}}
  {}{\typeout{Failed to patch title.}}
\makeatother

\usepackage{xpatch}
\makeatletter
\xapptocmd\appendix
  {\xapptocmd\section
    {\addcontentsline{toc}{section}{\appendixname\ifoneappendix\else~\theappendix\fi\\: #1}}
    {}{\InnerPatchFailed}%
  }
{}{\PatchFailed}
\keywords{ECG,Binocular-rivalry-paradigm,heart-rate,vision\newline\indent Word count: X}
\DeclareDelayedFloatFlavor{ThreePartTable}{table}
\DeclareDelayedFloatFlavor{lltable}{table}
\DeclareDelayedFloatFlavor*{longtable}{table}
\makeatletter
\renewcommand{\efloat@iwrite}[1]{\immediate\expandafter\protected@write\csname efloat@post#1\endcsname{}}
\makeatother
\usepackage{lineno}

\linenumbers
\usepackage{csquotes}
\ifLuaTeX
  \usepackage{selnolig}  % disable illegal ligatures
\fi
\IfFileExists{bookmark.sty}{\usepackage{bookmark}}{\usepackage{hyperref}}
\IfFileExists{xurl.sty}{\usepackage{xurl}}{} % add URL line breaks if available
\urlstyle{same} % disable monospaced font for URLs
\hypersetup{
  pdftitle={ECG\_Binocular\_Rivalry\_Paradigm},
  pdfauthor={Fan Gao1},
  pdflang={en-EN},
  pdfkeywords={ECG,Binocular-rivalry-paradigm,heart-rate,vision},
  hidelinks,
  pdfcreator={LaTeX via pandoc}}

\title{ECG\_Binocular\_Rivalry\_Paradigm}
\author{Fan Gao\textsuperscript{1}}
\date{}


\shorttitle{ECG\_study}

\authornote{

For this apa 6 style pdf document, I used Tinytex {[}\url{https://github.com/rstudio/tinytex-releases}{]}.

The authors made the following contributions. Fan Gao: Data collection, Writing - Original Draft Preparation, Writing - Review \& Editing.

Correspondence concerning this article should be addressed to Fan Gao, University of Chicago. E-mail: \href{mailto:fgao38@uchicago.com}{\nolinkurl{fgao38@uchicago.com}}

}

\affiliation{\vspace{0.5cm}\textsuperscript{1} University of Chicago}

\abstract{%
Though we are unconscious of most bodily sensations (e.g.~immune system), in a place where the internal (i.e.~self) and external (i.e.~physical world) interact, interoceptive stimuli---the sensation that arises from an internal organ (e.g.~heartbeat), have been found to yield an unexpected influence over how we see and sense the world (i.e.~exteroceptive stimuli). A substantial prior study has been dedicated to exploring how external stimuli affect our body and brain. For example, intentionally observing and recognizing external stimuli typically results in a deceleration of the heart rate, referred to as ``bradycardia of attention'' (Lacey, Kagan, Lacey, \& Moss, 1963). Such an effect is further examined in a follow-up study that showed subjects' heart rate decreased following a ready signal (Lacey \& Lacey, 1978). These findings have provided us with a novel understanding of how exteroceptive stimuli (e.g.~a ready signal at a traffic light) influence our interoceptive stimuli (e.g.~heart rate), but also raises the interesting question about the reverse effect: could interoceptive stimuli have an influence on exteroceptive stimuli? The question may seem counterintuitive at first since most of the interoceptive stimuli within one's self are not accessible (e.g.~immune system, heartbeat). For example, studies have suggested that only a quarter of the participants could perceive and judge their heart rate that closely synchronized with external stimuli above chance (Brener \& Ring, 2016). How can these interoceptive stimuli affect our perception of the world if we, for the most of time, do not have conscious access to them? Yet, recent research has shed light on this question.
}



\begin{document}
\maketitle

\begin{verbatim}
## [1] 5
## [1] "Hi"
##       names values
## tom     tom      1
## david david      2
## sam     sam      3
\end{verbatim}

\hypertarget{hello_world-function}{%
\subsubsection{\texorpdfstring{\emph{Hello\_world} function}{Hello\_world function}}\label{hello_world-function}}

{[}@{]}

\begin{verbatim}
## [1] "Today is: Monday Yes! I have no classes today"
## [1] "Today is: Tuesday Hello Dr. Dowling! I hope you are having a great day!"
## [1] "Today is: Wednesday Yes! I have no classes today"
## [1] "Today is: Thursday Hello Dr. Dowling! I hope you are having a great day!"
## [1] "Today is: Friday Yes! I have no classes today"
## [1] "Today is: Saturday Yes! I have no classes today"
## [1] "Today is: Sunday Yes! I have no classes today"
## [1] "Today is: Monday Yes! I have no classes today"
## [1] "Today is: Tuesday Good afternoon Dr.Hamilton"
## [1] "Today is: Wednesday Yes! I have no classes today"
## [1] "Today is: Thursday Good afternoon Dr.Hamilton"
## [1] "Today is: Friday Yes! I have no classes today"
## [1] "Today is: Saturday Yes! I have no classes today"
## [1] "Today is: Sunday Yes! I have no classes today"
## [1] "Today is: Monday Yes! I have no classes today"
## [1] "Today is: Tuesday Good afternoon Dr.Wang"
## [1] "Today is: Wednesday Yes! I have no classes today"
## [1] "Today is: Thursday Good afternoon Dr.Wang"
## [1] "Today is: Friday Yes! I have no classes today"
## [1] "Today is: Saturday Yes! I have no classes today"
## [1] "Today is: Sunday Yes! I have no classes today"
## [1] "Today is: Monday Yes! I have no classes today"
## [1] "Today is: Tuesday Yes! I have no classes today"
## [1] "Today is: Wednesday Yes! I have no classes today"
## [1] "Today is: Thursday Yes! I have no classes today"
## [1] "Today is: Friday Yes! I have no classes today"
## [1] "Today is: Saturday Yes! I have no classes today"
## [1] "Today is: Sunday Yes! I have no classes today"
\end{verbatim}

\hypertarget{introduction}{%
\section{Introduction}\label{introduction}}

Though we are unconscious of most bodily sensations (e.g.~immune system), in a place where the internal (i.e.~self) and external (i.e.~physical world) interact, interoceptive stimuli---the sensation that arises from an internal organ (e.g.~heartbeat), have been found to yield an unexpected influence over how we see and sense the world (i.e.~exteroceptive stimuli).\\
A substantial prior study has been dedicated to exploring how external stimuli affect our body and brain. For example, intentionally observing and recognizing external stimuli typically results in a deceleration of the heart rate, referred to as ``bradycardia of attention'' (Lacey, Kagan, Lacey, \& Moss, 1963). Such an effect is further examined in a follow-up study that showed subjects' heart rate decreased following a ready signal (Lacey \& Lacey, 1978). These findings have provided us with a novel understanding of how exteroceptive stimuli (e.g.~a ready signal at a traffic light) influence our interoceptive stimuli (e.g.~heart rate), but also raise the interesting question about the reverse effect: could interoceptive stimuli have an influence on exteroceptive stimuli? The question may seem counterintuitive at first since most of the interoceptive stimuli within one's self are not accessible (e.g.~immune system, heartbeat). For example, studies have suggested that only a quarter of the participants can perceive and judge their heart rate that is closely synchronized with external stimuli above chance (Brener \& Ring, 2016). How can these interoceptive stimuli affect our perception of the world if we, for the most of time, do not have conscious access to them? Yet, recent research has shed light on this question.
For instance, the heart rate and gastrointestinal tract (GI) are shown to be continuously producing electrical activity, thus sending messages to the brain, and eventually altering our perception and cognition (Azzalini, Rebollo, \& Tallon-Baudry, 2019). In addition, it is suggested that interoceptive stimuli may play a significant role in shaping emotions and cognition, and this process is derived from a low-level function---homeostatic regulation (Smith, 2017). These emotional states can, in turn, affect how we perceive the world. For instance, we may perceive a neutral stimulus as threatening when we are in an anxious state.
One study investigating the subjective experience of body ownership (EBO) presented even more compelling evidence of interaction between interoceptive and exteroceptive stimuli by showing that during the induction of a ``fake'' rubber hand, participants exhibited an increased sense of EBO if the heartbeat were synchronized with the color change of the rubber hand (Suzuki, Garfinkel, Critchley, \& Seth, 2013). Another study delves into examining how activation of certain cortical areas impacts subjects' hits and misses on a visual signal detection task: participants were asked to identify whether or not they saw a faint annulus; the study showed that the activation of ventromedial prefrontal cortex bilaterally (vACC-vmPFC), the site known for receiving cardiac inputs, were more likely to have participants consciously perceive the faint annulus (Park, Correia, Ducorps, \& Tallon-Baudry, 2014).
To our knowledge, despite these prior investigations on interoceptive and exteroceptive interaction, there is limited research that closely examines this effect visually. Also, there is still a notable gap in the existing literature, particularly in the context of visual bistable perceptual switching. Visual bistable perceptual switching refers to presenting participants with two visual stimuli, each of which dominates the visual field for a short period of time. This is usually achieved by using the binocular rivalry paradigm (Carmel, Arcaro, Kastner, \& Hasson, 2010). While prior studies have primarily investigated the realm of detection thresholds (A binary response: whether the signal or not), it is important to study this effect more comprehensively in a bistable perception. When perception oscillates between two ambiguous stimuli, it suggests dynamic processes at play in the brain. Understanding the mechanism of how the brain suppresses these perceptual ambiguities can shed light on fundamental aspects of perception and consciousness.
To fill in the gap, our research plans to use a binocular rivalry paradigm (Carmel et al., 2010), where one of two competing visual stimuli will be synchronized with the subjects' heartbeat (i.e.~electrocardiogram ECG signals) in real-time. Our goal is to investigate whether the synchronization of interoceptive stimuli (i.e.~heartbeat) will influence the prioritization of the visual stimuli in conscious awareness. Based on earlier studies that examined the effect of interoceptive stimuli on the brain (@ Azzalini et al., 2019) and homeostasis regulation (Smith, Thayer, Khalsa, \& Lane, 2017), we hypothesized to find that the visual stimulus that matched with the participant's real-time heartbeat should overall dominate the visual field longer than the stimulus that was not synchronized. In addition, we also expect to see that this effect is not dependent on participants' conscious awareness of their heartbeat sensations.

\hypertarget{methods}{%
\section{Methods}\label{methods}}

\textbf{Our experiment is going to be divided into two parts.}

\begin{enumerate}
\def\labelenumi{\arabic{enumi}.}
\item
  In the first section, we are planning to use a binocular rivalry paradigm -- presenting different visual stimuli, one to each eye of the participant; because the brain cannot process two visual stimuli simultaneously, one visual stimulus will dominate the other visual stimulus, see Figure 1. The idea is to synchronize one of the visual stimuli with the participant's heartbeat (measured by using an electrocardiogram ECG) in real-time; the synchronization of the heartbeat and visual stimulus is randomized, see Figure 2.Participants are not going to be told that one of the stimuli was synchronized with their real-time ECG; the Participants will identify which visual stimulus they are currently viewing by pressing the left (red) and right (blue) arrow keys.
\item
  In the second section, we are going to measure whether the participants could judge the external stimulus that is synchronized with their own heartbeat correctly. This will be done by presenting two pulsing circles, one synchronizes with the participant's ECG (immediately followed at the R peak) and the other one does not (followed later after the R peak).
\end{enumerate}

\hypertarget{participants}{%
\subsection{Participants}\label{participants}}

We aim to collect 60 undergraduate students taking Psychology courses at the University of Chicago. We are going to recruit participants through an online platform named SONA (Psychological and Brain Science Research System). Participants will need to have normal color vision and see well without glasses, as well as consent to participate in our study. Our participants' sample may not be representative since our sample consists of only college students, specifically students who are taking introductory Psychology courses. The introductory Psychology courses include a diverse population of students with different majors and backgrounds, but it is biased toward college and well-educated students at University of Chicago. However, as mentioned above, we do not expect that our results will vary significantly across races and genders since this effect is mostly driven by biological factors within the body. We are going to send our study protocol to the University of Chicago institutional review board for approval.

\hypertarget{material}{%
\subsection{Material}\label{material}}

\begin{itemize}
\item
  Electrocardiogram (ECG) was acquired at a 100 Hz sampling rate using a TMSi SAGA amplifier (TMSi, Netherlands)
\item
  ECG data was implemented in Python (see Data and Code Availability) using Lab Streaming Layer (LSL, labstreaminglayer.org)
\item
  ECG data were bandpass filtered to 5-15 Hz, and R-peaks were detected using the Pan-Thompkins algorithm (Pan \& Tompkins, 1985), modified from an existing implementation for LabGraph compatibility (Sznajder \& Łukowska, 2017)
\end{itemize}

\hypertarget{procedure}{%
\subsection{Procedure}\label{procedure}}

\hypertarget{data-analysis}{%
\subsection{Data analysis}\label{data-analysis}}

We used R {[}Version 4.2.2; (\textbf{R-base?}){]} for all our analyses.

\hypertarget{results}{%
\section{Results}\label{results}}

\hypertarget{discussion}{%
\section{Discussion}\label{discussion}}

\newpage

\hypertarget{references}{%
\section{References}\label{references}}

r

\hypertarget{refs}{}
\begin{CSLReferences}{1}{0}
\leavevmode\vadjust pre{\hypertarget{ref-azzalini_visceral_2019}{}}%
Azzalini, D., Rebollo, I., \& Tallon-Baudry, C. (2019). Visceral {Signals} {Shape} {Brain} {Dynamics} and {Cognition}. \emph{Trends in Cognitive Sciences}, \emph{23}(6), 488--509. \url{https://doi.org/10.1016/j.tics.2019.03.007}

\leavevmode\vadjust pre{\hypertarget{ref-brener_towards_2016}{}}%
Brener, J., \& Ring, C. (2016). Towards a psychophysics of interoceptive processes: The measurement of heartbeat detection. \emph{Philosophical Transactions of the Royal Society B: Biological Sciences}, \emph{371}(1708), 20160015. \url{https://doi.org/10.1098/rstb.2016.0015}

\leavevmode\vadjust pre{\hypertarget{ref-carmel_how_2010}{}}%
Carmel, D., Arcaro, M., Kastner, S., \& Hasson, U. (2010). How to {Create} and {Use} {Binocular} {Rivalry}. \emph{Journal of Visualized Experiments}, (45), 2030. \url{https://doi.org/10.3791/2030}

\leavevmode\vadjust pre{\hypertarget{ref-park_spontaneous_2014}{}}%
Park, H.-D., Correia, S., Ducorps, A., \& Tallon-Baudry, C. (2014). Spontaneous fluctuations in neural responses to heartbeats predict visual detection. \emph{Nature Neuroscience}, \emph{17}(4), 612--618. \url{https://doi.org/10.1038/nn.3671}

\leavevmode\vadjust pre{\hypertarget{ref-ryan_intergration_2017}{}}%
Smith, R., Thayer, J. F., Khalsa, S. S., \& Lane, R. D. (2017). The hierarchical basis of neurovisceral integration. \emph{Neuroscience and Biobehavioral Reviews}, \emph{75}, 274--296. \url{https://doi.org/10.1016/j.neubiorev.2017.02.003}

\leavevmode\vadjust pre{\hypertarget{ref-suzuki_multisensory_2013}{}}%
Suzuki, K., Garfinkel, S. N., Critchley, H. D., \& Seth, A. K. (2013). Multisensory integration across exteroceptive and interoceptive domains modulates self-experience in the rubber-hand illusion. \emph{Neuropsychologia}, \emph{51}(13), 2909--2917. \url{https://doi.org/10.1016/j.neuropsychologia.2013.08.014}

\end{CSLReferences}


\end{document}
